\documentclass[twoside,a4paper,11pt]{article}

\usepackage[T1]{fontenc}
\usepackage[utf8x]{inputenc}
\usepackage[ps2pdf]{hyperref}
\usepackage{reqlist}
\usepackage{longtable}

\newcommand{\urlink}[1]{\texttt{<\url{#1}>}}
\newcommand{\unix}{\textsc{Unix}}

\title{Dolda Connect protocol}
\author{Fredrik Tolf\\\texttt{<fredrik@dolda2000.com>}}

\begin{document}

\maketitle

\tableofcontents

\section{Introduction}
Dolda Connect consists partly of a daemon (a.k.a. server) that runs in
the background and carries out all the actual work, and a number of
client programs (a.k.a. user interfaces) that connect to the daemon in
order to tell it what to do. In order for the daemon and the clients
to be able to talk to each other, a protocol is needed. This document
intends to document that protocol, so that third parties can write
their own client programs.

It is worthy of note that there exists a library, called
\texttt{libdcui}, that carries out much of the low level work of
speaking the protocol, facilitating the creation of new client
programs. In itself, \texttt{libdcui} is written in the C programming
language and is intended to be used by other programs written in C,
but there also exist wrapper libraries for both GNU Guile (the GNU
project's Scheme interpreter) and for Python. The former is
distributed with the main Dolda Connect source tree, while the latter
is distributed separately (for technical reasons). To get a copy,
please refer to Dolda Connect's homepage:

\urlink{http://www.dolda2000.com/~fredrik/doldaconnect/}

\section{Transport format}
Note: Everything covered in this section is handled by the
\texttt{libdcui} library. Thus, if you read this because you just want
to write a client, and are using the library (or any of the wrapper
libraries), you can safely skip over this section. It may still be
interesting to read in order to understand the semantics of the
protocol, however.

The protocol can be spoken over any channel that features a
byte-oriented, reliable virtual (or not) circuit. Usually, it is
spoken over a TCP connection or a byte-oriented \unix\ socket. The
usual port number for TCP connections is 1500, but any port could be
used\footnote{However, port 1500 is what the \texttt{libdcui} library
  uses if no port is explicitly stated, so it is probably to be
  preferred}.

\subsection{Informal description}

On top of the provided byte-oriented connection, the most basic level
of the protocol is a stream of Unicode characters, encoded with
UTF-8. The Unicode stream is then grouped in two levels: lines
consisting of words (a.k.a. tokens). Lines are separated by CRLF
sequences (\emph{not} just CR or LF), and words are separated by
whitespace. Both whitespace and CRLFs can be quoted, however,
overriding their normal interpretation of separators and allowing them
to be parts of words. NUL characters are not allowed to be transferred
at all, but all other Unicode codepoints are allowed.

Lines transmitted from the daemon to the client are slightly
different, however. They all start with a three-digit code, followed
by either a space or a dash\footnote{Yes, this is inspired by FTP and
  SMTP.}, followed by the normal sequence of words. The three-digit
code identifies that type of line. Overall, the protocol is a
lock-step protocol, where the clients sends one line that is
interpreted as a request, and the daemon replies with one or more
lines. In a multi-line response, all lines except the last have the
three-digit code followed by a dash. The last line of a multi-line
response and the only line of a single-line response have the
three-digit code followed by a space. All lines of a multi-line
response have the same three-digit code. The client is not allowed to
send another request until the last line of the previous response has
been received. The exception is that the daemon might send (but only
if the client has requested it to do so) sporadic lines of
asynchronous notification messages. Notification message lines are
distinguished by having their three-digit codes always begin with the
digit 6. Otherwise, the first digit of the three-digit code indicates
the overall success or failure of a request. Codes beginning with 2
indicate the the request to which they belong succeeded. Codes
beginning with 3 indicate that the request succeeded in itself, but
that it is considered part of a sequence of commands, and that the
sequence still requires additional interaction before considered
successful. Codes beginning with 5 are indication of errors. The
remaining two digits merely distinguish between different
outcomes. Note that notification message lines may come at \emph{any}
time, even in the middle of multiline responses (though not in the
middle of another line). There are no multiline notifications.

The act of connecting to the daemon is itself considered a request,
solicitating a success or failure response, so it is the daemon that
first transmits actual data. A failure response may be provoked by a
client connecting from a prohibited source.

Quoting of special characters in words may be done in two ways. First,
the backslash character escapes any special interpretation of the
character that comes after it, no matter where or what the following
character is (it is not required even to be a special
character). Thus, the only way to include a backslash in a word is to
escape it with another backslash. Second, any interpretation of
whitespace may be escaped using the citation mark character (only the
ASCII one, U+0022 -- not any other Unicode quotes), by enclosing a
string containing whitespace in citation marks. (Note that the citation
marks need not necessarily be placed at the word boundaries, so the
string ``\texttt{a"b c"d}'' is parsed as a single word ``\texttt{ab
  cd}''.) Technically, this dual layer of quoting may seem like a
liability when implementing the protocol, but it is quite convenient
when talking directly to the daemon with a program such as
\texttt{telnet}.

\subsection{Formal description}

Formally, the syntax of the protocol may be defined with the following
BNF rules. Note that they all operate on Unicode characters, not bytes.

\begin{longtable}{lcl}
<session> & ::= & <SYN> <response> \\
 & & | <session> <transaction> \\
 & & | <session> <notification> \\
<transaction> & ::= & <request> <response> \\
<request> & ::= & <line> \\
<response> & ::= & <resp-line-last> \\
 & & | <resp-line-not-last> <response> \\
 & & | <notification> <response> \\
<resp-line-last> & ::= & <resp-code> <SPACE> <line> \\
<resp-line-not-last> & ::= & <resp-code> <DASH> <line> \\
<notification> & ::= & <notification-code> <SPACE> <line> \\
<resp-code> & ::= & ``\texttt{2}'' <digit> <digit> \\
 & & | ``\texttt{3}'' <digit> <digit> \\
 & & | ``\texttt{5}'' <digit> <digit> \\
<notification-code> & ::= & ``\texttt{6}'' <digit> <digit> \\
<line> & ::= & <CRLF> \\
 & & | <word> <ws> <line> \\
<word> & ::= & <COMMON-CHAR> \\
 & & | ``\texttt{$\backslash$}'' <CHAR> \\
 & & | ``\texttt{"}'' <quoted-word> ``\texttt{"}'' \\
 & & | <word> <word> \\
<quoted-word> & ::= & ``'' \\
 & & | <COMMON-CHAR> <quoted-word> \\
 & & | <ws> <quoted-word> \\
 & & | ``\texttt{$\backslash$}'' <CHAR> <quoted-word> \\
<ws> & ::= & <1ws> | <1ws> <ws> \\
<1ws> & ::= & <SPACE> | <TAB> \\
<digit> & ::= & ``\texttt{0}'' |
``\texttt{1}'' | ``\texttt{2}'' |
``\texttt{3}'' | ``\texttt{4}'' \\
& & | ``\texttt{5}'' | ``\texttt{6}'' |
``\texttt{7}'' | ``\texttt{8}'' |
``\texttt{9}''
\end{longtable}

As for the terminal symbols, <SPACE> is U+0020, <TAB> is U+0009,
<CRLF> is the sequence of U+000D and U+000A, <DASH> is U+002D, <CHAR>
is any Unicode character except U+0000, <COMMON-CHAR> is any
Unicode character except U+0000, U+0009, U+000A, U+000D, U+0020,
U+0022 and U+005C, and <SYN> is the out-of-band message that
establishes the communication channel\footnote{This means that the
  communication channel must support such a message. For example, raw
  RS-232 would be hard to support.}. The following constraints also
apply:
\begin{itemize}
\item <SYN> and <request> must be sent from the client to the daemon.
\item <response> and <notification> must be sent from the daemon to
  the client.
\end{itemize}
Note that the definition of <word> means that the only way to
represent an empty word is by a pair of citation marks.

In each request line, there should be at least one word, but it is not
considered a syntax error if there is not. The first word in each
request line is considered the name of the command to be carried out
by the daemon. An empty line is a valid request as such, but since no
matching command, it will provoke the same kind of error response as
if a request with any other non-existing command were sent. Any
remaining words on the line are considered arguments to the command.

\section{Data model}

The main purpose of the protocol is to communicate the current state
of the daemon to the client and keep it synchronized. Therefore, in
order to understand the actions of the individual requests, an
understanding of the data structures that define the current state is
fundamental. The intent of this section is document those structures
in a top-down approach.

\subsection{Filesharing network}
\label{fnet}
At the heart of the Dolda Connect daemon lies the abstraction of a
file sharing network, often abbreviated ``filenet'' or ``fnet''. To
the daemon, a filenet is a software module that speaks a certain
filesharing protocol, such as the Direct Connect protocol. A client
program will never interact directly with any filenet module, but it
is often important to know that there are several filenet
modules\footnote{Actually, at the time of this writing, that is false,
  as only the Direct Connect protocol is implemented. However, the
  protocol still requires it explicitly stated at several occasions,
  and it is nonetheless important to keep in mind that there
  \emph{could} be several filenet modules. Also, work is under way to
  implement ADC, the ``official'' successor to the Direct Connect
  protocol.}. The only detail visible to clients about a filenet is
its name. The currently implemented filenet modules are listed in
section \ref{fnets}, along with important information about each.

\subsection{Filenet node}
\label{fnetnode}
The filenet node, often abbreviated ``fnetnode'', corresponds closely
to the Direct Connect concept of a ``hub''.

\section{Requests}

For each arriving request, the daemon checks so that the request
passes a number of tests before carrying it out. First, it matches the
name of the command against the list of known commands to see if the
request calls a valid command. If the command is not valid, the daemon
sends a reponse with code 500. Then, it checks so that the request has
the minimum required number of parameters for the given command. If it
does not, it responds with a 501 code. Last, it checks so that the
user account issuing the request has the necessary permissions to have
the request carried out. If it does not, it responds with a 502
code. After that, any responses are individual to the command in
question. The intention of this section is to list them all.

\subsection{Permissions}

As for the permissions mentioned above, it is outside the scope of
this document to describe the administration of
permissions\footnote{Please see the \texttt{doldacond.conf(5)} man
  page for more information on that topic.}, but some commands require
certain permission, they need at least be specified. When a connection
is established, it is associated with no permissions. At that point,
only requests that do not require any permissions can be successfully
issued. Normally, the first thing a client would do is to authenticate
to the daemon. At the end of a successful authentication, the daemon
associates the proper permissions with the connection over which
authentication took place. The possible permissions are listed in
table \ref{tab:perm}.

\begin{table}
  \begin{tabular}{rl}
    Name & General description \\
    \hline
    \texttt{admin} & Required for all commands that administer the
    daemon. \\
    \texttt{fnetctl} & Required for all commands that alter the state of
    connected hubs. \\
    \texttt{trans} & Required for all commands that alter the state of
    file transfers. \\
    \texttt{transcu} & Required specifically for cancelling uploads. \\
    \texttt{chat} & Required for exchanging chat messages. \\
    \texttt{srch} & Required for issuing and querying searches. \\
  \end{tabular}
  \caption{The list of available permissions}
  \label{tab:perm}
\end{table}

\subsection{Protocol revisions}
\label{rev}
Since Dolda Connect is developing, its command set may change
occasionally. Sometimes new commands are added, sometimes commands
change argument syntax, and sometimes commands are removed. In order
for clients to be able to cleanly cope with such changes, the protocol
is revisioned. When a client connects to the daemon, the daemon
indicates in the first response it sends the range of protocol
revisions it supports, and each command listed below specifies the
revision number from which its current specification is valid. A
client should should check the revision range from the daemon so that
it includes the revision that incorporates all commands that it wishes
to use.

Whenever the protocol changes at all, it is given a new revision
number. If the entire protocol is backwards compatible with the
previous version, the revision range sent by the server is updated to
extend forward to the new revision. If the protocol in any way is not
compatible with the previous revision, the revision range is moved
entirely to the new revision. Therefore, a client can check for a
certain revision and be sure that everything it wants is supported by
the daemon.

At the time of this writing, the latest protocol revision is 2. Please
see the file \texttt{doc/protorev} that comes with the Dolda Connect
source tree for a full list of revisions and what changed between
them.

\subsection{List of commands}

Follows does a (hopefully) exhaustive listing of all commands valid
for a request. For each possible request, it includes the name of the
command for the request, the permissions required, the syntax for the
entire request line, and the possible responses.

The syntax of the request and response lines is described in a format
like that traditional of \unix\ man pages, with a number of terms,
each corresponding to a word in the line. Each term in the syntax
description is either a literal string, written in lower case; an
argument, written in uppercase and meant to be replaced by some other
text as described; an optional term, enclosed in brackets
(``\texttt{[}'' and ``\texttt{]}''); or a list of alternatives,
enclosed in braces (``\texttt{\{}'' and ``\texttt{\}}'') and separated
by pipes (``\texttt{|}''). Possible repetition of a term is indicated
by three dots (``\texttt{...}''), and, for the purpose of repition,
terms may be groups with parentheses (``\texttt{(}'' and
``\texttt{)}'').

Two things should be noted regarding the responses. First, in the
syntax description of responses, the response code is given as the
first term, even though it is not actually considered a word. Second,
more words may follow after the specified syntax, and should be
discarded by a client. Many responses use that to include a human
readable string to indicate the conclusion of the request.

\subsubsection{Connection}
As mentioned above, the act of connecting to the daemon is itself
considered a request, soliciting a response. Such a request obviously
has no command name and no syntax, but needs a description
nonetheless.

\revision{1}

\noperm

\begin{responses}
  \response{200}
  The old response given by daemons not yet using the revisioned
  protocol. Clients receiving this response should consider it an
  error.
  \response{201 LOREV HIREV}
  Indicates that the connection is accepted. The \param{LOREV} and
  \param{HIREV} parameters specify the range of supported protocol
  revisions, as described in section \ref{rev}.
  \response{502 REASON}
  The connection is refused by the daemon and will be closed. The
  \param{REASON} parameter states the reason for the refusal in
  English\footnote{So it is probably not suitable for localized
    programs}.
\end{responses}

\input{commands}

\section{Filesharing networks}
\label{fnets}

\end{document}
